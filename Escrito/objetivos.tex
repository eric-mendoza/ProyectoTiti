%  % % % % % % % % % % % % % % % % % % % % % % % % % % % % % % % % % % % %
%
%                                                               Objetivos
% % % % % % % % % % % % % % % % % % % % % % % % % % % % % % % % % % % % %

\Chapter{Objetivos}

A continuaci�n se presentan los objetivos que se persegu�an alcanzar con el desarrollo del trabajo. 

\section{Generales}
\begin{itemize}
\item Establecer la relaci�n que existe entre la profesi�n deseada durante la etapa tentativa y la carrera escogida de los estudiantes de la Universidad del Valle de Guatemala.
\end{itemize}

\section{Espec�ficos}
\begin{itemize}
\item Identificar las limitaciones que tienen las profesiones de la etapa tentativa de los estudiantes
\item Cuantificar el porcentaje de alumnos que estudian la profesi�n que tuvieron durante la etapa tentativa
\item Enumerar factores afectan la toma decisi�n de una carrera universitaria
\item Describir el concepto tiene la sociedad acerca de las profesiones deseadas por los j�venes durante la etapa tentativa en general
\item Definir la relaci�n que existe entre el g�nero del estudiante y su carrera de etapa tentativa
\item Listar las carreras de etapa tentativa m�s comunes entre los estudiantes de la Universidad del Valle de Guatemala

\end{itemize}