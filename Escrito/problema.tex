%  % % % % % % % % % % % % % % % % % % % % % % % % % % % % % % % % % % % %
%
%                                                               Planteamiento del problema
% % % % % % % % % % % % % % % % % % % % % % % % % % % % % % % % % % % % %

\Chapter{Planteamiento del Problema}
\section{Enunciado del Problema}

Durante el tiempo que se ha estado en la Universidad del Valle de Guatemala se notado un fen\'{o}meno en varios estudiantes, el cual es la inconformidad parcial o total con la carrera escogida. Dicha inconformidad se ha visto reflejada en el mal desempe\~{n}o actitudinal, como acad\'{e}mico de los estudiantes en los cursos inherentes a la carrera escogida o al observar que muchos deciden cambiar de carrera  escogida al principio de sus estudios. Por lo que se piensa que la carrera que cursan no necesariamente es la deseada para su futuro profesional.\\

Con base en lo anterior, se ha decidido formular la siguiente pregunta de investigaci�n: \textquestiondown Cu\'{a}l es la relaci\'{o}n entre la profesi�n deseada durante la etapa tentativa y la carrera universitaria escogida de los estudiantes de la \textbf{Universidad del Valle de Guatemala}?\\

Asimismo, este trabajo de investigaci�n se encontrar� guiado por las siguientes preguntas auxiliares: 

\begin{itemize}
	\item \textquestiondown Qu\'{e} porcentaje de alumnos estudian la carrera universitaria que tuvieron durante la etapa tentativa?
	\item \textquestiondown Qu\'{e} factores afectan la elecci�n de una carrera universitaria?
	\item \textquestiondown Cu�les son las profesiones de etapa tentativa m�s comunes entre los estudiantes de la Universidad del Valle de Guatemala?
	\item \textquestiondown Qu\'{e} relaci\'{o}n existe entre el g\'{e}nero del estudiante y su carrera deseada durante la etapa tentativa?
\end{itemize}

\subsection{Causas} Dicho fen\'{o}meno podr\'{\i}a estar siendo causado por varios factores, dentro de los cuales se podr\'{\i}an incluir:
\begin{itemize}
	\item No haber seguido la carrera deseada, debido a que socialmente se tiene un mal concepto de las personas que desempe\~{n}an dicha profesi\'{o}n. 
	\item Padres en contra de la carrera deseada por el estudiante, quit\'{a}ndole todo el apoyo moral por seguirla. 
	\item La mala remuneraci\'{o}n que se obtendr\'{\i}a al desempe\~{n}ar dicha profesi\'{o}n. O bien el poco campo para trabajar con la profesi\'{o}n deseada.
\end{itemize}

\subsection{Consecuencias:} Las consecuencias incluyen:
\begin{itemize}
	\item Falta de motivaci\'{o}n al estudiar la carrera que el estudiante decidi\'{o} cursar en lugar de la que deseaba. 
	\item Frustraci\'{o}n.
	\item Infelicidad.
	\item Mediocridad en desempe\~{n}o acad\'{e}mico o bien suma dependencia de otros compa\~{n}eros para lograr cursar la carrera.
	\item A largo plazo una serie de profesionales que posiblemente no desempe\~{n}ar\'{a}n su profesi\'{o}n como es debido, puesto que a lo que se dedican no les produce satisfacci\'{o}n. 
\end{itemize}

\subsection{Indicadores:} Algunas se\~{n}ales que muestran esta inconformidad o infelicidad son:
\begin{itemize}
	\item La falta de motivaci\'{o}n en ciertos cursos de la carrera, aunque el curso sea propio de la carrera.
	\item Indicadores de subempleo  demostrando la poca aplicaci\'{o}n de ciertas profesiones.
	\item Cantidad de estudiantes de cada carrera.
	\item Mal desempe\~{n}o acad\'{e}mico.
	\item Inter\'{e}s relativamente alto por cursos poco trascendentales en el enfoque de la carrera.
	\item Leve comunicaci\'{o}n entre estudiantes de la carrera estudiada debido a escasos intereses en com\'{u}n.
	\item Mayor afinidad con estudiantes de carreras diferentes a la escogida.
\end{itemize}

%
%\section{Formulaci\'{o}n del problema}
%\subsection{Pregunta central} 
%\textquestiondown Cu\'{a}l es la relaci\'{o}n entre la profesi�n deseada durante la etapa tentativa y la carrera universitaria escogida de los estudiantes de la \textbf{Universidad del Valle de Guatemala}?\\

