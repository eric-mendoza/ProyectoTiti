%  % % % % % % % % % % % % % % % % % % % % % % % % % % % % % % % % % % % %
%
%                                                               Planteamiento del problema
% % % % % % % % % % % % % % % % % % % % % % % % % % % % % % % % % % % % %

\Chapter{Planteamiento del Problema}
\begin{document}
	content...\textbf{Enunciado del Problema}\\
	
	\textbf{Descripción}\\
	
	Durante el tiempo que se ha estado en la Universidad del Valle de Guatemala se notado un fenómeno en varios estudiantes, el cual es la inconformidad parcial o total con la carrera escogida. Dicha inconformidad se ha visto reflejada en el mal desempeño actitudinal, como académico de los estudiantes en los cursos inherentes a la carrera escogida o al observar que muchos deciden cambiar de carrera  escogida al principio de sus estudios. Por lo que se piensa que la carrera que cursan no necesariamente es la deseada para su futuro profesional.\\
	
	\textbf{Causas:} Dicho fenómeno podría estar siendo causado por varios factores, dentro de los cuales se podrían incluir: 
	●	No haber seguido la carrera deseada, debido a que socialmente se tiene un mal concepto de las personas que desempeñan dicha profesión. 
	●	Padres en contra de la carrera deseada por el estudiante, quitándole todo el apoyo moral por seguirla. 
	●	La mala remuneración que se obtendría al desempeñar dicha profesión. O bien el poco campo para trabajar con la profesión deseada.\\
	
	\textbf{Consecuencias:} Las consecuencias incluyen: 
	●	Falta de motivación al estudiar la carrera que el estudiante decidió cursar en lugar de la que deseaba. 
	●	Frustración 
	●	Infelicidad.
	●	Mediocridad en desempeño académico o bien suma dependencia de otros compañeros para lograr cursar la carrera.
	●	A largo plazo una serie de profesionales que posiblemente no desempeñarán su profesión como es debido, puesto que a lo que se dedican no les produce satisfacción. \\
	
	
	\textbf{Indicadores:} Algunas señales que muestran esta inconformidad o infelicidad son: 
	●	La falta de motivación en ciertos cursos de la carrera, aun el curso sea propio de la carrera. 
	●	Indicadores de subempleo  demostrando la poca aplicación de ciertas profesiones. 
	●	Cantidad de estudiantes de cada carrera.
	●	Mal desempeño académico.
	●	Interés relativamente alto por cursos poco trascendentales en el enfoque de la carrera.
	●	Leve comunicación entre estudiantes de la carrera estudiada debido a escasos intereses en común.
	●	Mayor afinidad con estudiantes de carreras diferentes a la escogida. \\
	
	\textbf{Formulación del problema}\\
	
	\textbf{Pregunta central:} ¿Cuál es la relación entre la profesión de ensueño y la carrera universitaria de los estudiantes de la Universidad del Valle de Guatemala?\\
	
	\textbf{Preguntas auxiliares:}
	●	¿Cuales son las limitaciones que tienen las profesiones de ensueño de los estudiantes?
	●	¿Qué porcentaje de alumnos tienen como profesión de ensueño la carrera universitaria que estudia?
	●	¿Qué factores afectan la toma decisión de una carrera universitaria?
	●	¿Qué concepto tiene la sociedad acerca de las profesiones de ensueño de los jóvenes en general?
	●	¿Qué relación existe entre el género del estudiante y su profesión de ensueño?
	●	¿Cuales son las profesiones de ensueño más comunes entre los estudiantes de la Universidad del Valle de Guatemala?
	
\end{document}
