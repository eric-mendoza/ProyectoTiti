%  % % % % % % % % % % % % % % % % % % % % % % % % % % % % % % % % % % % %
%
%                                                               Recomendaciones
% % % % % % % % % % % % % % % % % % % % % % % % % % % % % % % % % % % % %

\Chapter{Recomendaciones}
Se deja como recomendaci�n los siguientes puntos:\\

\begin{itemize}
\item Tomar un curso sobre �tica y los pasos que se deben de seguir al realizar investigaci�n que involucra a sujetos humanos. y para futuras generaciones se debe considerar siempre someter todos los protocolos de trabajo de graduaci�n a un comit� de �tica desde el inicio del trabajo. \\

\item Realizar un plan desde el inicio de la investigaci�n para la comunicaci�n con los hospitales a estudiar. Esto se qued� como lecci�n por parte del grupo ya que la investigaci�n se vio retrasada en un punto por la gesti�n de todos los permisos de entrada y validaci�n de datos.
\item Completar la implementaci�n de todos los m�dulos generados. Debido a la informaci�n recopilada en los tres hospitales p�blicos, donde se realiz� este trabajo, fue posible observar que la necesidad de tecnolog�a en esta �rea es grande. Por lo tanto, se exhorta a la continuaci�n de este proyecto para cualquier hospital p�blico o privado.\\

\item Entrevistar equitativamente a todos los perfiles que se definen en la metodolog�a de \emph{design thinking}. De esta forma es posible evitar cualquier sesgo en la cantidad de necesidades que surgen durante las entrevistas. \\

\end{itemize}
