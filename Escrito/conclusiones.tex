%  % % % % % % % % % % % % % % % % % % % % % % % % % % % % % % % % % % % %
%
%                                                              Conclusiones
% % % % % % % % % % % % % % % % % % % % % % % % % % % % % % % % % % % % %
\Chapter{Conclusiones}

\begin{itemize}

\item Es claro que existen ya millones de herramientas que han sido creadas para el �rea de salud, pero \emph{�cu�ntas de estas se adaptan al sector p�blico de Guatemala?}. La innovaci�n de este trabajo fue establecer el dise�o del sistema en base a las necesidades identificadas directamente en Guatemala. 

\item A partir de la investigaci�n realizada hasta la fase de exploraci�n, fue posible encontrar las cinco �reas de necesidad que presentaban los hospital en estudio.

\item Se gener� un prototipo refinado seg�n la metodolog�a de \emph{design thinking}, haciendo uso de la tecnolog�a necesaria para mejorar la atenci�n que reciben los pacientes en los hospitales p�blicos de Guatemala. Todo ello a partir del recorrido completo de la metodolog�a durante un per�odo de 10 meses.

\item Se dise�� un sistema de informaci�n, que establece una comunicaci�n que va desde el hospital hacia el paciente. Utilizando como canal de comunicaci�n entre ellos el servicio de mensajer�a corta el cual todos los tel�fonos m�viles poseen. 

\item Se dise�� un sistema de informaci�n, que establece una comunicaci�n que va desde el hospital hacia el paciente. Utilizando como canal de comunicaci�n entre ellos el servicio de mensajer�a corta el cual todos los tel�fonos m�viles poseen. 

\item El presente trabajo no s�lo propone al �rea de salud p�blica de Guatemala tecnolog�as de bajos recursos, sino que provee a la comunidad de desarrolladores el dise�o de un prototipo que integra las plataformas OpenMRS y FrontlineSMS.
\end{itemize}