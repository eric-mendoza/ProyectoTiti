%  % % % % % % % % % % % % % % % % % % % % % % % % % % % % % % % % % % % %
%
%                                                               Marco Te�rico
% % % % % % % % % % % % % % % % % % % % % % % % % % % % % % % % % % % % %

\Chapter{Marco Te�rico}

Despu�s de una explicaci�n del por qu� la realizaci�n del trabajo y a lo que se espera llegar, se hace la transici�n a la siguiente fase te�rica. Esta secci�n marca el inicio de los resultados de investigaci�n previos al desarrollo de la tercera fase, es decir, el trabajo pr�ctico. Su prop�sito general es brindarle al lector los conocimientos base para comprender todas las herramientas utilizadas. En general, se establecen tres componentes principales: (1) el sistema de salud, (2) teor�a de las metodolog�as y (3) tecnolog�as. 

% Sistema hospitalario nacional
\section{Secci�n1}

No se que vamos a poner aca\\

\subsection{SubSeccion1}
Aqui van los subtemas

\begin{enumerate}
	\item Asi se ponen listas
	\item Numero 2


\subsection{SubSeccion2}
Subtema dos



\subsection{Financiamiento}
ya no se que poner\\

\subsubsection{Hola soy un sub sub tema}
Este es un sub-sub-tema\\


% Historia Cl�nica
\section{Secci�n2}
No se que ir� aca\\

\subsection{Subseccion1}
Y as� nos vamos\\