%  % % % % % % % % % % % % % % % % % % % % % % % % % % % % % % % % % % % %
%
%                                                               Variables
% % % % % % % % % % % % % % % % % % % % % % % % % % % % % % % % % % % % %
\Chapter{Variables}
\section{Definici�n conceptual de variables}
\begin{itemize}
\item Etapa tentativa:  lapso de tiempo entre los 11 y 18 a�os de una persona. En esta etapa las personas comienzan a reconocer las actividades en las que pueden desenvolverse de mejor manera, por lo que es la antesala a la etapa donde se concretan las decisiones que le dan rumbo a la vida de una persona. (Ginzber et al, 1951. En Pereira, 1997)
	
\item	Profesi�n de etapa tentativa: Se define como un empleo, facultad u oficio que se desea entre los 11 y 18 a�os.

\item Carrera  universitaria:  Conjunto de estudios realizados en la universidad que habilitan para el ejercicio de una profesi�n.
\end{itemize}
\section{Definici�n operacional de variables:}
\begin{itemize}
\item Carrera universitaria: Todas las carreras de la Universidad del Valle de Guatemala.
\item Carrera tentativa: conjunto de carreras que hicieron una respectiva ilusi�n o fantas�a a los estudiantes de la Universidad del Valle de Guatemala
\end{itemize}	
\begin{table}[htb]
\begin{center}
	\begin{tabular}{|l|l|l|l|}
		\hline
		Variable & Origen & Escala de medici�n & Relaci�n \\
		\hline \hline
		Carrera universitaria & Cualitativo & Categ�rica/Nominal & Dependiente \\ \hline
		Carrera tentativa & Cualitativo & Categ�rica/Nominal & Independiente \\ \hline
		Influencia de los padres & Cualitativo & Categ�rica/Nominal & Interviniente \\ \hline
		Popularidad & Cuantitativo & N�merica/Discreta & Interviniente \\ \hline
		Situaci�n econ�mica & Cualitativo & Categ�rica/Nominal & Interviniente \\ \hline
		Edad & Cuantitativo & Num�rica/Discreta & Confusora \\ \hline
		\hline
		Sexo & Cuantitativo & Num�rica/Discreta & Confusora \\ \hline
	\end{tabular}
	\caption{Clasificaci�n de variables}
	\label{tabla:sencilla}
\end{center}
\end{table}



