%  % % % % % % % % % % % % % % % % % % % % % % % % % % % % % % % % % % % %
%
%                                                               Hip�tesis
% % % % % % % % % % % % % % % % % % % % % % % % % % % % % % % % % % % % %

\Chapter{Hip�tesis}

A continuaci�n se presentan las hip�tesis planteadas en la investigaci�n para el desarrollo de la misma. 
\section{Hip�tesis de investigaci�n}
\begin{itemize}
	\item Existe una correlaci�n estad�sticamente significativa de 0.50 seg�n la escala de Pearson entre  la profesi�n deseada durante la etapa tentativa y la carrera escogida de los estudiantes de la Universidad del Valle de Guatemala.
\end{itemize}

\section{Hip�tesis nula}
\begin{itemize}
	\item No existe una correlaci�n estadisticamente  significativa entre  la profesi�n de ensue�o durante la etapa tentativa y la carrera escogida de los estudiantes de la Universidad del Valle de Guatemala.
\end{itemize}

\section{Hip�tesis alternativas}
\begin{itemize}
	\item Existe una corrrelaci�n estadisticamente significativa de 0.50 seg�n la escala de Pearson entre  la influencia de los padres o tutores del estudiante y la carrera escogida por los estudiantes de la Universidad del Valle de Guatemala.
	\item Existe una corrrelaci�n estadisticamente significativa de 0.50 seg�n la escala de Pearson entre  la  situaci�n economica del estudiante y la carrera escogida por los estudiantes de la Universidad del Valle de Guatemala. 
	\item Existe una corrrelaci�n estadisticamente significativa de 0.50 seg�n la escala de Pearson entre  la popularidad de ciertas carreras y la carrera escogida de los estudiantes por la Universidad del Valle de Guatemala.
	
\end{itemize}

