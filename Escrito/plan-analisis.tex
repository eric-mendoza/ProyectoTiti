\Chapter{Plan de An�lisis}
\section{Unidad de An�lisis}

\subsection{Relaci�n entre la carrera tentativa y la actual.}

La pregunta es la siguiente: \textquestiondown Considera usted que existe una relaci�n entre la carrera que quer�a estudiar previo a la Universidad y la carrera que se encuentra estudiando en la actualidad? \\

Con base en esta pregunta, es posible recopilar informaci�n importante y pertinente sobre la relaci�n entre la carrera tentativa y la actual. Primero, es posible conocer el tipo de relaci�n entre ambas carreras, es decir si la relaci�n es directa o indirecta. Esto se refiere a que el estudiante podr�a encontrarse estudiando una carrera universitaria debido a que esta misma conforma la carrera de ensue�o, o bien, podr�a estar estudiando algo que no tiene relaci�n en lo absoluto. Conocer la opini�n del estudiante sobre la relaci�n entre ambas carreras sirve para determinar en qu� medida el estudiante es capaz de crear una conexi�n entre las mismas. Esto es de gran ayuda, ya que ayuda a comprender las razones por las cuales el estudiante piensa que lo que est� estudiando actualmente le servir� para constru�r la carrera de sus sue�os, sino ser�a imposible tratar de entender las razones por las que el estudiante se encuentra en dicha carrera. Si el estudiante responde directamente que no existe una relaci�n, esto significa que �l mismo sabe que no podr� encaminar su carrera actual a su carrera de ensue�o y es prec�samente esto lo que importa en este trabajo de investigaci�n. \\

Con todo esto, lo que se quiere explicar es que es posible recopilar entonces el n�mero de personas que encuentran una relaci�n entre carreras y el n�mero de personas que no encuentran una relaci�n, lo que ayudar� a analizar la medida en la que un estudiante de la Universidad del Valle de Guatemala se encuentra estudiando una carrera por que le encuentra sentido y uso en su vida. Ser� posible, de hecho, encontrar una proporci�n, la cual servir� de an�lisis estad�stico. Dentro de esta proporci�n (alumnos que encuentran relaci�n y alumnos que no), se encuentra otra informaci�n importante. Para los alumnos que encuentran una relaci�n entre la carrera tentativa y la actual, existen dos posibilidades. La primera es que la carrera tentativa sea exactamente la misma que la carrera actual, por lo que es clara la relaci�n: al ser ambas carreras la misma, el estudiante est� cumpliendo con sus espectativas de estudiar algo que le gusta. La segunda posibilidad es que las carreras sean distintas. Si esto sucede, entonces la conexi�n que el estudiante encuentre entre ambas carreras es el factor interesante a estudiar. Aqu� nuevamente es posible realizar un an�lisis estad�stico y encontrar las proporciones correspondientes. 