\Chapter{Plan de An�lisis}
\section{Unidad de An�lisis}

\subsection{Preguntas de la encuesta}
\begin{itemize}
	\item Edad: Con la primera pregunta lo que se desea obtener es la edad que el encuestado tiene. Con gr�ficas de barras se podr�a mostrar c�mo est�n distribuidas las personas encuestadas en relaci�n a su edad, es decir que tantas personas de cierta edad fueron entrevistadas.\\
	
	\item Sexo: Con la segunda pregunta lo que se desea obtener es el sexo de la persona que fue entrevistada. Con este dato se pueden realizar cuadros de contingencia entre el sexo de las personas y la categor�a a la que pertenece la carrera tentativa del estudiante.\\
	
	\item Carrera: El objetivo de preguntarle al encuestado la carrera que cursa actualmente en la universidad es para posteriormente realizar una correlaci�n entre su carrera tentativa y la actual. De ese modo se espera poder encontrar la validez de la hip�tesis de investigaci�n\\
	
	\item Cursos generales: El objetivo de esta pregunta es poder es realizar comparaciones entre la satisfacci�n del estudiante con la carrera universitaria y si este tambi�n escogi� su carrera.\\
	

	\item Cursos Espec�ficos: Al igual que con la pregunta anterior, el objetivo de esta es realizar comparaciones entre la satisfacci�n del estudiante con la carrera universitaria y si este tambi�n escogi� su carrera\\

	\item Satisfacci�n de la carrera: Con esta pregunta se busca poder realizar comparaciones con la respuesta a la pregunta de si el estudiante est� estudiando su carrera tentativa.\\

	
	
\item Carrera deseada durante etapa tentativa: Con esta pregunta se busca recopilar todas las carreras/profesiones que fueron consideradas por los estudiantes de la muestra antes de tomar la decisi�n definitiva. Esto con el fin de dar respuesta a la interrogante de qu� carreras tentativas son m�s comunes entre los estudiantes y si existe una correlaci�n entre la carrera tentativa y la universitaria.\\

		
\item La decisi�n de su carrera: Esta pregunta nos permite saber qu� proporci�n de los estudiantes de la muestra tuvieron la oportunidad de escoger la carrera que estudian actualmente. Esto debido a que existen casos en los que los estudiantes son forzados a estudiar cierta carrera por parte de sus encargados o padres. Con esta informaci�n se puede verificar la validez de la hip�tesis alternativa planteada.\\


\item Relaci�n entre carreras: Esta pregunta permite encontrar qu� proporci�n de los estudiantes s� encontraban una relaci�n entre la carrera tentativa y la universitaria.\\
	
\item Factores que influyeron en la elecci�n de la carrera: Con esta pregunta se busca encontrar la influencia que tuvieron ciertos factores en la elecci�n de carrera de los estudiantes de la Universidad del Valle de Guatemala\\

	
	\item Influencia de los padres o personas encargadas: Con esta pregunta se busca obtener la informaci�n necesaria para realizar la correlaci�n entre la posibilidad del estudiante para escoger su carrera universitaria y la influencia de los padres o encargados en ellos. Con esto se puede evaluar la validez de la hip�tesis alternativa.\\
	
	
	\item Deseo de cambiarse de carrera: En esta pregunta se desea medir las intenciones del estudiante de realizar un cambio de carrera, con el debido an�lisis se encuentra relacionada con la pregunta de satisfacci�n del estudiante con su carrera actual, ya que pueda que algunos estudiantes no se encuentren satisfechos con su carrera, pero que no tengan deseos de cambiarse, probablemente por factores que se lo impiden. Debido a esto lo que se busca es encontrar una relaci�n entre la satisfacci�n del estudiante con su carrera actual y las intenciones que este tiene de cambiarse de carrera.\\
\end{itemize}

 