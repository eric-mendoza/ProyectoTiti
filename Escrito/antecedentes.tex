%  % % % % % % % % % % % % % % % % % % % % % % % % % % % % % % % % % % % %
%
%                                                               Antecedentes
% % % % % % % % % % % % % % % % % % % % % % % % % % % % % % % % % % % % %

\Chapter{Antecedentes}
Diversos estudios han tratado de analizar los factores que inciden en la elecci\'{o}n de carreras por parte de las personas, tomando en cuenta aspectos propios de las personas que generan atracci\'{o}n o inclinaci\'{o}n hacia cierto campo o \'{a}rea profesional, dando como resultado interesantes observaciones y an\'{a}lisis que permitir\'{a}n guiar de mejor forma este trabajo de investigaci\'{o}n. Resulta interesante, entonces, conocer teor\'{\i}as como la de \textbf{Roe, Holland, Super y Ginzberg, Ginsburg, Axelrad y Herma}, con el fin de tener un concepto previo m\'{a}s enriquecido antes de proseguir con el proceso investigativo. \\

\section{Teor\'{\i}a de Roe sobre la influencia de la personalidad en la elecci\'{o}n de carreras}
Esta teor\'{\i}a sostiene que cada individuo hereda una tendencia a gastar sus energ\'{\i}as de una manera particular (Roe, 1964. En Osipow, 1990). Al combinarse esta manera de consumir la energ\'{\i}a con diferentes experiencias durante el transcurso de vida de la persona, se desarrolla una modo de cumplir con sus necesidades. Este modo de cumplir con sus necesidades incide en el comportamiento de elecci\'{o}n de carrera. Seg\'{u}n Osipow (1990), La teor\'{\i}a de Roe intenta presentar de manera expl\'{\i}cita las relaciones entre los factores gen\'{e}ticos y la conducta vocacional.\\

Para desarrollar dicha teor\'{\i}a Roe realiz\'{o} estudios que se enfocaron en dos aspectos en general (Osipow, 1990). El primero de ellos consisti\'{o} en la realizar pruebas de habilidades administrativas y proyectivas a un grupo de cient\'{\i}ficos. Con base en los resultados de dicho estudio, Roe encontr\'{o} que existen diferencias y similitudes entre los cient\'{\i}ficos de diferentes campos. El segundo aspecto consisti\'{o} en el estudio de los antecedentes de dichos científicos. Para ello realiz\'{o} entrevistas y pruebas en los que se tocaron temas como la infancia, el desarrollo psicosocial, religiosidad y experiencias laborales.\\

\section{Teor\'{\i}a de las carreras de Holland}
Seg\'{u}n Pereira (1997), Holland basa su teor\'{\i}a en la psicolog\'{\i}a diferencial, la cual observa a los individuos como producto de herencia y del ambiente. Dicha teor\'{\i}a dice que \textit{las personas pueden ser categorizadas en una de los siguientes tipos de personalidad: realista, investigador, social, convencional, emprendedor y artista, los cuales corresponden a los siguientes tipos de ambiente de trabajo: pr\'{a}ctico, de investigaci\'{o}n, social, convencional, de empresa y art\'{\i}stico} (Holland, 1971. En Pereira, 1997).\\

Pereira (1997) comenta que Holland toma en cuenta el tipo de personalidad de la persona y lo relaciona con los diferentes tipos de trabajo mencionados anteriormente. Y que las personas que trabajan con la misma ocupaci\'{o}n tienden a tener una personalidad similar, lo cual implica una historia de desarrollo personal similar. \\

La teoría de Holland es similar a la teoría de Roe, ya que ambos toman como un factor influyente la personalidad de la persona. En lo que las teor\'{\i}as divergen es en la manera en la que se desarrolla la personalidad del individuo. Para Roe la personalidad es el resultado de herencia, mientras que para Holland la personalidad depende del ambiente del individuo durante su desarrollo.\\

\section{Teor\'{\i}a de Ginzberg, Ginsburg, Axelrad y Herma}
Ginzber y sus compa\~{n}eros (Ginzber et al, 1951. En Pereira, 1997) catalogaron el desarrollo vocal como el resultado de la interacci\'{o}n de las siguientes variables: la realidad, la cantidad y calidad de educaci\'{o}n, los factores emocionales y de personalidad y los valores del individuo. \\

Para ellos la elecci\'{o}n vocacional era un proceso de varias elecciones en el transcurso del desarrollo personal de cada individuo. El cual finalizaba con la elecci\'{o}n de una vocaci\'{o}n. Este proceso lo dividieron en tres periodos (Pereira, 1997). El primero es la fantas\'{\i}a, etapa que termina a los 11 a\~{n}os. En este per\'{\i}odo se ignoran todos los factores para la selecci\'{o}n de vocaci\'{o}n, los ni\~{n}os fantasean con lo que quieren llegar a ser. El segundo es el tentativo, de 11 a 18 a\~{n}os. En este lapso de tiempo los individuos comienzan a reconocer las actividades en las que mejor se desenvuelven. Esta etapa es una antesala al \'{u}ltimo periodo. El \'{u}ltimo es el realista, de 18 a 24 a\~{n}os. En esta etapa se escoge el camino que se seguir\'{a} y se fundamentan metas a largo plazo. \\

En comparaci\'{o}n con las dos teor\'{\i}as analizadas previamente, esta teor\'{\i}a se diferencia en que la decisi\'{o}n de vocaci\'{o}n no ocurre como un \'{u}nico paso final, sino que se toma como un proceso. En las anteriores la decisi\'{o}n vocacional es un influido por la personalidad de la persona influido por las circunstancias que lo rodean.\\

\section{Teor\'{\i}a de Donald Super}
La teor\'{\i}a de Super se basa en la definici\'{o}n de \textquotedblleft orientaci\'{o}n vocacional\textquotedblright, el cual es:
\textit{El proceso de ayudar a la persona a desarrollar y aceptar una imagen integrada y adecuada de s\'{\i} mismo y de su rol en el mundo del trabajo, comprobar este concepto frente a la realidad y convertirla en realidad, con satisfacci\'{o}n para s\'{\i} mismo y para la sociedad} (Super, 1951. En Alzina, 1996).\\

La teor\'{\i}a de Super tiene un \'{e}nfasis en la influencia psicol\'{o}gica de la orientaci\'{o}n vocacional, ya que mezcla la dimensi\'{o}n personal y vocacional de la orientaci\'{o}n. Por ello, en esta teor\'{\i}a el individuo es considerado como un todo (Alzina, 1996).\\

Al igual que Ginzberg y sus colegas, Super toma la vocaci\'{o}n como un proceso. La diferencia entre estas dos teor\'{\i}as radica en que Super indica que las personas asumen diferentes papeles a lo largo de su vida, cambiando tambi\'{e}n los estilos de vida. Seg\'{u}n Super (1951. En S\'{a}nchez, 2014), \textit{la elecci\'{o}n vocacional es el producto de todas las experiencias en los diferentes papeles que se toman en el desarrollo personal del individuo.} Los roles que las personas toman se pueden clasificar en las siguientes etapas: crecimiento, exploraci\'{o}n, establecimiento, sostenimiento y declive.