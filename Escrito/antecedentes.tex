%  % % % % % % % % % % % % % % % % % % % % % % % % % % % % % % % % % % % %
%
%                                                               Antecedentes
% % % % % % % % % % % % % % % % % % % % % % % % % % % % % % % % % % % % %

\Chapter{Antecedentes}

\begin{document}
	Diversos estudios han tratado de analizar los factores que inciden en la elección de carreras por parte de las personas, tomando en cuenta aspectos propios de las personas que generan atracción o inclinación hacia cierto campo o área profesional, dando como resultado interesantes observaciones y análisis que permitirán guiar de mejor forma este trabajo de investigación. Resulta interesante, entonces, conocer teorías como la de Roe, Holland, Super y Ginzberg, Ginsburg, Axelrad y Herma, con el fin de tener un concepto previo más enriquecido antes de proseguir con el proceso investigativo. \\
	
	\textbf{Teoría de Roe sobre la influencia de la personalidad en la elección de carreras}\\
	
	Esta teoría sostiene que cada individuo hereda una tendencia a gastar sus energías de una manera particular (Roe, 1964. En Osipow, 1990). Al combinarse esta manera de consumir la energía con diferentes experiencias durante el transcurso de vida de la persona, se desarrolla una modo de cumplir con sus necesidades. Este modo de cumplir con sus necesidades incide en el comportamiento de elección de carrera. Según Osipow (1990), La teoría de Roe intenta presentar de manera explícita las relaciones entre los factores genéticos y la conducta vocacional.\\
	
	Para desarrollar dicha teoría Roe realizó estudios que se enfocaron en dos aspectos en general (Osipow, 1990). El primero de ellos consistió en la realizar pruebas de habilidades administrativas y proyectivas a un grupo de científicos. En base a los resultados de dicho estudio, encontró que existen diferencias y similitudes entre los científicos de diferentes campos. El segundo aspecto consistió en el estudio de los antecedentes de dichos científicos. Para ello realizó entrevistas y pruebas en los que se tocaron temas como la infancia, el desarrollo psicosocial, religiosidad y experiencias laborales.\\
	
	\textbf{Teoría de las carreras de Holland}\\
	
	Según Pereira (1997), Holland basa su teoría en la psicología diferencial, la cual observa a los individuos como producto de herencia y del ambiente. Dicha teoría dice que \textit{las personas pueden ser categorizadas en una de los siguientes tipos de personalidad: realista, investigador, social, convencional, emprendedor y artista, los cuales corresponden a los siguientes tipos de ambiente de trabajo: práctico, de investigación, social, convencional, de empresa y artístico} (Holland, 1971. En Pereira, 1997).\\
	
	Pereira (1997) comenta que Holland toma en cuenta el tipo de personalidad de la persona y lo relaciona con los diferentes tipos de trabajo mencionados anteriormente. Y que las personas que trabajan con la misma ocupación tienden a tener una personalidad similar, lo cual implica una historia de desarrollo personal similar.
	La teoría de Holland es similar a la teoría de Roe, en que ambos toman como un factor influyente la personalidad de la persona. En lo que las teorías divergen es en la manera en la que se desarrolla la personalidad del individuo. Para Roe la personalidad es el resultado de herencia, mientras que para Holland la personalidad depende del ambiente del individuo durante su desarrollo.\\
	
	\textbf{Teoría de Ginzberg, Ginsburg, Axelrad y Herma}\\
	
	Ginzber y sus compañeros (Ginzber et al, 1951. En Pereira, 1997) catalogaron el desarrollo vocal como el resultado de la interacción de las siguientes variables: la realidad, la cantidad y calidad de educación, los factores emocionales y de personalidad y los valores del individuo.
	
	Para ellos la elección vocacional era un proceso de varias elecciones en el transcurso del desarrollo personal de cada individuo. El cual finalizaba con la elección de una vocación. Este proceso lo dividieron en tres periodos (Pereira, 1997). El primero es la fantasía, etapa que termina a los 11 años. En este periodo se ignoran todos los factores para la selección de vocación, los niños fantasean con lo que quieren llegar a ser. El segundo es el tentativo, de 11 a 18 años. En este lapso de tiempo los individuos comienzan a reconocer las actividades en las que mejor se desenvuelven. Esta etapa es una antesala al último periodo. El último es el realista, de 18 a 24 años. En esta etapa se escoge el camino que se seguirá y se fundamentan metas a largo plazo.
	
	En comparación con las dos teorías analizadas previamente, esta teoría se diferencia en que la decisión de vocación no ocurre como un único paso final, sino que se toma como un proceso. En las anteriores la decisión vocacional es un influido por la personalidad de la persona influido por las circunstancias que lo rodean.\\
	
	\textbf{Teoría de Donald Super}\\
	
	La teoría de Super se basa en la definición de “orientación vocacional”, el cual es:
	\textit{El proceso de ayudar a la persona a desarrollar y aceptar una imagen integrada y adecuada de sí mismo y de su rol en el mundo del trabajo, comprobar este concepto frente a la realidad y convertirla en realidad, con satisfacción para sí mismo y para la sociedad} (Super, 1951. En Alzina, 1996).\\
	
	La teoría de Super tiene un énfasis en la influencia psicológica de la orientación vocacional, ya que mezcla la dimensión personal y vocacional de la orientación. Por ello, en esta teoría el individuo es considerado como un todo (Alzina, 1996).\\
	
	Al igual que Ginzberg y sus colegas, Super toma la vocación como un proceso. La diferencia entre estas dos teorías radica en que Super indica que las personas asumen diferentes papeles a lo largo de su vida, cambiando también los estilos de vida. Según Super (1951. En Sánchez, 2014), \textit{la elección vocacional es el producto de todas las experiencias en los diferentes papeles que se toman en el desarrollo personal del individuo.} Los roles que las personas toman se pueden clasificar en las siguientes etapas: crecimiento, exploración, establecimiento, sostenimiento y declive.
\end{document}


