%  % % % % % % % % % % % % % % % % % % % % % % % % % % % % % % % % % % % %
%
%                                                               Justificaci�n
% % % % % % % % % % % % % % % % % % % % % % % % % % % % % % % % % % % % %

\Chapter{Justificaci�n}
Este trabajo de investigaci�n surge gracias a las diversas observaciones realizadas en la Universidad del Valle de Guatemala, en donde la mayor�a de estudiantes muestran tener un concepto de "profesi�n de ensue�o" que desear�an que existiese en sus vidas, sin embargo no lo pueden poner en practica debido a la presencia de distintos factores que dificultan la realizaci�n de este concepto mismo. Es posible que debido a ello, dichos estudiantes se vean en la necesidad de estudiar una carrera universitaria que no tenga relaci�n alguna con su profesi�n de ensue�o, lo que resulta muchas veces en que las personas se ocupen en un �rea que no les apasiona. Sin embargo, esto no sucede siempre. Existe tambi�n el caso en el que algunos estudiantes encuentran la relaci�n entre su carrera universitaria y su profesi�n de ensue�o, lo que hace que su ocupaci�n sea un �xito total en cuanto a la satisfacci�n que les genera ocuparse en algo que les gusta. Asimismo, puede darse el caso que existan alumnos en los cuales su concepto de "profesi�n de ensue�o" se encuentre inherente a la carrera que han elegido.\\